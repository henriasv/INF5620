% Eksamensoppgave i INF5620

\documentclass[a4paper, 10pt]{article}
\usepackage[utf8x]{inputenc}
\usepackage{cancel}
\usepackage{graphicx}
\usepackage{amsmath}
\newcommand{\mb}{\mathbf}
\newcommand{\mc}{\mathcal}
\newcommand{\n}{\nabla}

\author{Henrik Andersen Sveinsson}
\title{Eksamensoppgave 3 - INF5620}
\date{\today}


\begin{document}
\maketitle

\section{Oppgavetekst}
\subsection{a} 
Consider a vibration problem

\begin{equation}
	u''(t)+\omega^2 u(t)=0,u(0)=I, u'(0)=V
\end{equation}
Here, $\omega$, $I$, and $V$ are given data. We discretize the ODE by centered diferences,

\begin{equation}
	[D_tD_tu+\omega^2u=0]^n
\end{equation}
Explain how the stability and the accuracy of this scheme can be investigated via exact solutions of the discrete equations, and quote the main results. Illustrate the numerical problems that can arise from this scheme.

\subsection{b} 
We now consider a 1D wave equation
\begin{equation}
u_{tt}=c^2u_{xx}	
\end{equation}

with some appropriate boundary and initial conditions. Explain how the stability and accuracy of a centered difference scheme,

\begin{equation}
	[D_tD_tu=c^2D_xD_x u]^n_i
\end{equation}

can be investigated via exact solutions of the discrete equations. Quote the main results.

\subsection{c}
 Explain how the analysis can help us to understand why a smooth initial condition gives relatively small numerical artifacts, and why a less smooth initial condition gives rise to significant numerical artifacts. The movies below show a wave propagating with unit Courant number in a medium and the wave enters another medium with 1/4 of the wave velocity (implying a Courant number of 1/4 in that medium). The propagation of waves in the left medium is exact, while the propagation in the other medium is subject to numerical errors.

\subsection{d} 
Explain how a truncation error analysis is carried out for the problem in a). Find correction terms such that the order of the scheme becomes $\Delta t^4$.

\subsection{e} 
Explain how a truncation error analysis is carried out for the problem in b). Find correction terms such that the order of the scheme becomes $\Delta t^4,\Delta x^4$.

\section{Svingelikningen:}
Diskret likning:
\begin{equation}
	[D_t D_t u +\omega^2 u = 0 ]^n
\end{equation}

Gir et eksplisitt numerisk skjema:
\begin{equation}
	u^{n+1} = 2u^n - u^{n-1} - \Delta t^2\omega^2 u^n
\end{equation}
Siden differenslikningen er homogen med konstante koeffisienter forventer vi en løsning på formen
\begin{equation}
	u = IA^n
\end{equation}
Det er rimelig å anta at det finnes løsninger av denne diskrete likningen som likner på løsningen av de kontinuerlige likningene, som er   
\begin{equation}
	u_e = Ie^{i\omega t}
\end{equation}
Vi regner med at de diskrete likningene vil ha en lik løsning, men med endret $\omega$. Vi setter inn en $\tilde{\omega}$.

Dette betyr at vi prøver å finne løsninger på formen
\begin{equation}
	u^n  =IA^n = Ie^{\tilde{\omega}\Delta t n} = I e^{i\tilde{\omega}t}
\end{equation}

Vi setter dette inn i sentrert differanseoperator, og får:
\begin{equation}
	[D_t D_t u]^n = -I\exp{(i\tilde\omega t)}\frac{4}{\Delta t^2}\sin^2(\frac{\tilde\omega\Delta t}{2})
\end{equation}
Så setter vi dette inn i den diskrete likningen, og løser for  $\tilde{\omega}$
\begin{equation}
	-I e^{i\tilde{\omega}t}\frac{4}{\Delta t^2}\sin^2{\frac{\tilde{\omega}\Delta t}{2}} + \omega^2 Ie^{i\tilde{\omega}t} = 0
\end{equation}
Vi finner ut ifra dette, at for at påstanden om at den diskrete løsningen skal være $I e^{i\tilde{\omega}t}$, så må $\tilde{\omega}$ være gitt ved:
\begin{equation}
	\tilde{\omega} = \pm \frac{2}{\Delta t} \sin^{-1} \left(\frac{\omega\Delta t}{2}\right) \geq 1
\end{equation}
Dette tallet er alltid større enn 1, som betyr at svingefrekvensen til løsningen av den diskrete likningen er større enn svingefrekvensen til løsningen av den kontinuerlige likningen. For å analysere mer på denne likningen ser vi på Taylorutviklingen til 2. orden:
\begin{equation}
	\frac{\tilde\omega}{\omega} \approx 1 + \frac{1}{24}p^2
\end{equation}

For et lite antall tidssteg per svingning, vi lman få en ganske høy frekvens i den diskrete løsningen sammenliknet med den kontinuerlige. Det vil gi fullstendig faseforskyvning av løsningen etterhvert.

Stabilitetskriteriet på dette skjemaet er gitt ved at $\tilde{\omega}$ må være reell for at amplituden til løsningen ikke skal variere. Da må
\begin{equation}
	\frac{\omega\Delta t}{2} \in [-1, 1]
\end{equation}
Altså må
\begin{equation}
	|\omega \Delta t| \leq 2
\end{equation}
for at vi skal få en stabil løsning.

\begin{itemize}
\item Viktigste parameter for løsningene er $\omega \Delta t$
\item Antall tidssteg i hver periode er $N_p = \frac{2\pi }{\omega \Delta t}$
\item $p=\omega \Delta t \leq 2$ gir konstant amplitude på løsningen.
\item Det er en relativ fasefeil på $\tilde{\omega}/\omega \approx 1 + \frac{1}{24}p^2$ relativt til løsningen av den kontinuerlige likningen.  
\end{itemize}


\section{Bølgelikningen i 1D}
\begin{equation}
	u_{tt} = c^2 u_{xx} 
\end{equation}
En sentrert differanse ser slik ut:
\begin{equation}
	[D_t D_t u = c^2 D_x D_x u]_i^n
\end{equation}
Dette gir en formel for det neste tidssteget:
\begin{equation}
	u^{n+1}_i = -u^{n-1}_i + 2u^n_i + C^2
\left(u^{n}_{i+1}-2u^{n}_{i} + u^{n}_{i-1}\right)
\end{equation}
Der vi har definert Couranttallet $C = c\frac{\Delta t}{\Delta x}$

\subsection{Stabilitetsanalyse}
Vi vet at løsiningene av den kontinuerlige bølgelikningen er på formen $u(x, t) = e^{i(kx - \omega t)}$. Derfor ser vi etter løsninger av den diskrete likningen på formen 
\begin{equation}
	u_q^n = e^{i(kx_q - \tilde{\omega }t_n)}
\end{equation}

Dette er kun en bølgekomponent, så interessante løsninger av bølgelikninen må bygges opp av mange slike. I stabilitetsanalysen vil vi derfor kreve at løsningene skal være stabile for alle $k$ og $\omega$.

Vi begynner med å sette inn denne løsningen for de endelige differansene:
\begin{equation}
	[D_tD_t e^{i\omega t}]^n = -\frac{4}{\Delta t^2}\sin^2\left(
\frac{\omega\Delta t}{2}\right)e^{i\omega n\Delta t}
\end{equation}
\begin{equation}
	[D_xD_x e^{ikx}]_q = -\frac{4}{\Delta x^2}\sin^2\left(
\frac{k\Delta x}{2}\right)e^{ikq\Delta x} 
\end{equation}

Så settes disse sammen i den diskrete likningen:
\begin{equation}
	\frac{4}{\Delta t^2}\sin^2\left(\frac{\tilde\omega\Delta t}{2}\right)
= c^2 \frac{4}{\Delta x^2}\sin^2\left(\frac{k\Delta x}{2}\right)
\end{equation}
\begin{equation}
	\sin^2\left(\frac{\tilde\omega\Delta t}{2}\right)
= C^2\sin^2\left(\frac{k\Delta x}{2}\right)
\end{equation}
der $C = \frac{c\Delta t}{\Delta x}$


Herifra ønsker vi å gjøre to ting:
\begin{itemize}
\item Finne stabilitetskriteriet $C\leq 1$
\item finne fasefeilen $\tilde{\omega}/\omega$
\end{itemize}

Vi begynner med stabilitetskriteriet:
Vi tar roten av likningen, og får 
\begin{equation}
	\sin\left(\frac{\tilde\omega\Delta t}{2}\right)
= C\sin\left(\frac{k\Delta x}{2}\right)
\end{equation}
Siden $\omega$ er reell i den eksakte løsningen, er det naturlig å se etter reelle $\tilde{\omega}$ Dessuten er det den eneste løsningen som ikke divergerer. Dette skyldes at man vil få to komplekskonjugerte løsninger for $\tilde{\omega}$ om den er kompleks, hvorav en vil få løsningen til å vokse. 

Når vi har krevd en reell $\tilde{\omega}$, har vi også sagt at $C\sin{\frac{k\Delta x}{2}} \in [-1, 1]$. Siden sinusleddet kan anta hele $[-1, 0]$, så må $C\leq 1$.

Så tar vi den numeriske dispersjonsrelasjonen, som gir oss fasefeilen:
Vi løser likningen for $\tilde{\omega}$:
\begin{equation}
	\tilde\omega = \frac{2}{\Delta t}
\sin^{-1}\left( C\sin\left(\frac{k\Delta x}{2}\right)\right)
\end{equation}

Vi ser umiddelbart at for $C=1$ at $\tilde{\omega} = \frac{2}{\Delta t}\frac{k\Delta x}{2} = \omega$, slik at den numeriske løsningen er eksakt i meshpunktene. 

Det kan være nyttig å se på hvordan dette påvirker bølgehastighetene $\tilde{c} = \frac{\tilde\omega}{k}$, så vi plotter det. Dette plottet forklarer hvorfor vi gjerne ser artefakter med stygge initialbetingelser.

\begin{equation}
	\frac{\tilde{c}}{c} = \frac{1}{C p} \sin^{-1}(\sin p)
\end{equation}

\subsection{Konsekvenser for artefakter}
En faktisk initialbetingelse og løsning av bølgelikningen vil være en lineærkombinasjon av mange bølgekomponenter. Dersom man har et Couranttall som er $C<1$ vil korte bølger bevege seg saktere enn lange bølger. Dersom man har brukt mange bølgekomponenter for å bygge opp en form, vil denne formen kunne endre seg siden bølgene den er bygget opp av beveger seg i forskjellig hastigheter. Funksjoner som inneholder diskontinuiteter krever mange korte bølger for å representeres. For initialbetingelse som krever stort innslag av korte bølger, vil man se mye artefakter for store tidssteg. Dette ser man spesielt godt dersom man ser på bølger som skifter medium, siden det ene mediet da nødvendigvis må ha $C \neq 1$


\section{Korreksjonstermer så skjemaet for svingelikning blir $\Delta t^4$}
Trunkeringsfeilen er det tallet man får når man setter inn den eksakte løsningen $u_e$ i den diskrete likningen $\mc{L}_{\Delta}(u) = 0$: $R = \mc{L}_{\Delta}(u_e)$.  

Trunkeringsfeilanalysen gjøres ved å sette inn taylorutviklingen av de forskjellige leddene i en finite difference. 
For sentrert differanse $[D_tD_t u]^n$ er det relevant å se på:
\begin{equation}
	u(t_n) = u(t_n)
\end{equation}
\begin{equation}
	u(t_{n-1}) = u(t_n)-u'(t_n)\Delta t + \frac{1}{2} u''(t_n)\Delta t^2 - \frac{1}{6}u'''(t_n)\Delta t^3 + \frac{1}{24} u''''(t_n)\Delta t^4 + \mathcal{O}(t^5)
\end{equation}
\begin{equation}
	u(t_{n+1}) = u(t_n)+u'(t_n)\Delta t + \frac{1}{2} u''(t_n)\Delta t^2 + \frac{1}{6}u'''(t_n)\Delta t^3 + \frac{1}{24} u''''(t_n)\Delta t^4 + \mathcal{O}(t^5)
\end{equation}

Generelt kaller vi nå taylorekspansjonen av $u^n$ for $T_u^n$. Da gjøres trunkeringsanalysen som 
\begin{equation}
 	[D_tD_t u] = \frac{T_u^{n+1} - 2T_u^{n} + T_u^{n-1}}{\Delta t^2} = u''(t_n) + \frac{1}{12}u''''(t_n) \Delta t^2 + \mathcal{O}(\Delta t^4)
 \end{equation} 

Tanken med et korreksjonsledd er at vi kan legge til et ledd $C$ i likningen slik at
\begin{equation}
	[D_tD_t u_e + \omega^2u_e = C+R]^n
\end{equation}
Der $R$ er fjerde ordens.
Ved å derivere på den opprinnelige likningen får vi $u'''' = \omega^4 u$.
Korreksjonsledde vårt skal kansellere $\frac{1}{12}u''''(t_n)\Delta t^2$. Dermed blir det nye skjemaet
\begin{equation}
	[D_t D_t u + \omega^2(1 - \frac{1}{12}\omega^2\Delta t^2)u = 0]
\end{equation}
\section{Korreksjonstermer så skjemaet for bølgelikningen blir $\Delta x^4, \Delta t^4$}

Vi ønsker å gjøre tilsvarende for bølgelikningen. Vi tenker oss at vi vil ha et numerisk skjema 
\begin{equation}
	[D_t D_t u_e = c^2 D_xD_x u_e + C + R]^n_i
\end{equation}
Der $R = \mc{O}(\Delta t^4, \Delta x^4)$.

Fra å skrive ut de endelige differansene slik som i forrige oppgave, får vi at
\begin{equation}
	R^n_i + C^n_i = \frac{1}{12} u_{e, tttt}(x_i, t_n)\Delta t^2 - c^2\frac{1}{12} u_{e, xxxx} (x_i, t_n)\Delta x^2 + \mc{O}(\Delta t^4, \Delta x^4)
\end{equation}
For at skjemaet skal bli 4. ordens må
\begin{equation}
	C = \frac{1}{12} u_{e, tttt}(x_i, t_n)\Delta t^2 - c^2\frac{1}{12} u_{e, xxxx} (x_i, t_n)\Delta x^2
\end{equation}
Fra bølgelikningen vet vi at:
\begin{equation}
	u_{tttt} = c^2u_{xxtt}
\end{equation}
\begin{equation}
	u_{xxxx} = \frac{1}{c^2} u_{ttxx}
\end{equation}
Så antar vi at funksjonene er glatte nok til å skifte rekkefølge på de deriverte, slik at vi kan skrive at:
\begin{equation}
	\frac{1}{12} u_{e, tttt}(x_i, t_n)\Delta t^2 - c^2\frac{1}{12} u_{e, xxxx} (x_i, t_n)\Delta x^2 = \frac{1}{12} (c^2\Delta t^2 - \Delta x^2)[D_xD_xD_tD_tu]_i^n
\end{equation}

Det numeriske skjemaet på operatorform er da:
\begin{equation}
	[D_tD_tu = c^2 D_xD_x u + \frac{1}{12} (c^2\Delta t - \Delta x^2) D_xD_xD_tD_tu)]_i^n
\end{equation}

Så burde man vel egentlig sjekke at den diskretiseringen vi gjør av korreksjonsleddet også er av orden $\Delta t^4$.
\end{document}