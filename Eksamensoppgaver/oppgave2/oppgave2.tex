% Eksamensoppgave i INF5620

\documentclass[a4paper, 10pt]{article}
\usepackage[utf8x]{inputenc}
\usepackage{cancel}
\usepackage{graphicx}
\usepackage{amsmath}
\newcommand{\mb}{\mathbf}
\newcommand{\mc}{\mathcal}
\newcommand{\n}{\nabla}
\newcommand{\p}{\partial}

\author{Henrik Andersen Sveinsson}
\title{Eksamensoppgave 2 - INF5620}
\date{\today}


\begin{document}
\maketitle

\section{Oppgavetekst}

\subsection*{a}
 Set up a wave equation problem in 2D with zero normal derivative as boundary condition. Assume a variable wave velocity.

Mention a physical problem where this mathematical model arises. Explain the physical interpretation of the unknown function.
\subsection*{b}
 Present a finite difference discretization. Explain in particular how the boundary conditions and the initial conditions are incorporated in the scheme.
\subsection*{c}
 Explain (in princple) how the 2D discretization can be extended to 3D.
\subsection*{d}
 Set up the stability condition in 3D. Also quote results on about accuracy of the method in 3D and define the accuracy measure(s) precisely.
\subsection*{e}
 Explain how you can verify the implementation of the method.
\subsection*{f}
 The scheme for the wave equation is perfect for parallel computing. Why? What are the principal ideas behind a parallel version of the scheme? 

\section{Bølgelikningen på grunt vann i 2D}

\begin{equation}
	\frac{\p^2}{\p t^2} u(t, x, y) = \frac{\p}{\p x} \left(q(x, y) \frac{\p}{\p x} u\right) + \frac{\p}{\p y} \left(q(x, y) \frac{\p}{\p y} u\right)
\end{equation}

Denne likningen oppstår ved modellering av tsunamier. Jeg har antatt en konstant tetthet, slik at den kan inngå i funksjonen $q$. Den ukjente funksjonen $u$ er vannivået relativt til gjennomsnittlig vannivå. $q(x, y) = v(x, y)^2$ der $v$ er bølgehastigheten. Grensebetingelsen som vi vil sette opp er $\frac{\p u}{\p n} = 0$, som er en Neumannbetingelse.


\section{Finite difference diskretisering}

Her velger jeg å bruke tilfellet at $q$ er konstant, for å få litt mindre å skrive. Jeg tenker det er fornuftig når det kun er 15 minutter. 
(Spørre HPL om det er greit å anta dette her)
Da ser likningen slik ut:
\begin{equation}
	u_{tt} = c^2(u_{xx} + u_{yy})
\end{equation}

Vi velger en sentert differanse i alle ledd, med $\Delta x = \Delta y = h$

\begin{equation}
	\frac{u_{i, j}^{n+1} - 2u_{i,j}^n + u_{i, j}^{n-1}}{\Delta t^2} = c^2\frac{u_{i+1, j}^n - 2u_{i, j}^n + u_{i-1, j}^n + u_{i, j+1}^n - 2u_{i, j}^n + u_{i, j-1}^n}{\Delta h^2}
\end{equation}
\begin{equation}
	u_{i, j}^{n+1} = 2u_{i,j}^n - u_{i,j}^{n-1} + \frac{c^2 \Delta t^2}{h^2}\left( u_{i+1, j}^n + u_{i-1, j}^n + u_{i, j+1}^n + u_{i, j-1}^n - 4u_{i,j}^n\right)
\end{equation}
Dette git en metode for å finne tilstanden i et tidssteg, gitt alle funksjonsverdiene i de to forrige tidsstegene. Men vi ser at man kun finner funksjonsverdier for punkter (i, j), men for å finne disse trenger vi punktene rett utenfor randen. Her kommer randbetingelsene inn i bildet. Vi skal se på to typer grensebetingelser. For grunt vann-likninger er nok Neumannbetingelsen den mest relevante, men vi skal også se på Dirichletbetinglesen. 

\subsection{Dirichletbetingelsen}
Den enkleste randbetingelsen man kan tenke seg:
\begin{equation}
	u(\mb{x}) = f(\mb{x}) \forall \mb{x} \in \p\Omega
\end{equation}
Dette løser problemet vi hadde på randen. Dersom vi har en rand med kjente verdier trenger vi bare å regne ut verdiene strengt innenfor denne, og vi har dermed alltid tilgang til verdier som ligger ett hakk "utenfor" de verdiene vi regner ut. 

\subsection{Neumannbetingelsen}
Neumannbetingelsen går på den deriverte på randen, og vi setter $\frac{\p u}{\p n} = f(\mb{x})$ på $\p\Omega$. Dersom vi ser på områder som kun har normal i koordinatretningene, er det greit å takle disse randbetingelsene også, ved å gjøre en endelig differanse-tilnærming til randbetingelsen:
\begin{equation}
	\frac{u_{i+1}-u_{i-1}}{2h} = f(u_i)
\end{equation}
Dette gir 
\begin{equation}
	u_{i+1} = u_{i-1} + 2hf(u_i)
\end{equation}
For høyre og øvre grenseflate, og 
\begin{equation}
	u_{i-1} = u_{i+1} - 2hf(u_i) 
\end{equation}
For venste og nedre grenseflate. 
Her evaluerer man i prinsippet punkter som ligger utenfor domenet man jobber på, så dette legges inn før man regner funksjonsverdier på det indre av domenet.

\subsection{Initialbetingelse}
For å løse bølgelinkningen trenger man to initialbetingelser i hvert punkt: Funksjonsverdien og den første deriverte. Da kan man lage et "Ghost timestep" for tidssteg $-1$. 

Vi kjører en differansemetode på den deriverte av initialbetingelsen:
\begin{equation}
	\frac{u_{i, j}^{n+1}-u_{i, j}^{n-1}}{2\Delta t} = V_{i, j}
\end{equation}
\begin{equation}
	u^{-1}_{i,j} = u^1_{i,j} - 2\Delta t V_{i,j}
\end{equation}

Vi setter dette inn i det generelle skjemaet, og får for $u^1$
\begin{equation}
	u_{i, j}^{1} = 2u_{i,j}^0 - (u^1_{i,j} - 2\Delta t V_{i,j}) + \frac{c^2 \Delta t^2}{h^2}\left( u_{i+1, j}^0 + u_{i-1, j}^0 + u_{i, j+1}^0 + u_{i, j-1}^0 - 4u_{i,j}^0\right)
\end{equation}

Innsatt for initialbetingelse og snudd litt på, får vi:
\begin{equation}
	u_{i, j}^1 = I_{i,j} + 2\Delta t V_{i,j} + \frac{c^2\Delta t}{h^2} (I_{i+1, j} + I_{i-1, j} + I_{i, j+1}+I_{i, j-1} - 4I_{i,j})
\end{equation}
Her har vi eliminert dette "Ghost timestep" $-1$, så nå kan man regne ut $u^1$, og deretter kjøre algoritmen vanlig.


\section{Utvidelse til 3D}
Utvidelsen til 3D er nærmest triviell. Man legger på hele del-operatoren i likningen:
\begin{equation}
	u_{tt} = c^2 \nabla^2 u
\end{equation}

\section{Stabilitet i 3D}

\section{Verifikasjon}


\section{Parallellisering}
Skjemaet er perfekt for parallellisering, fordi man vet alt man trenger for neste tidssteg direkte i et tidssteg, og fordi man kun trenger informasjon for noder i nærheten for å regne oppførsel i neste tidssteg. 

\end{document}