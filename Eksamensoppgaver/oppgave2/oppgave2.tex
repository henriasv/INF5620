% Eksamensoppgave i INF5620

\documentclass[a4paper, 10pt]{article}
\usepackage[utf8x]{inputenc}
\usepackage{cancel}
\usepackage{graphicx}
\usepackage{amsmath}
\newcommand{\mb}{\mathbf}
\newcommand{\mc}{\mathcal}
\newcommand{\n}{\nabla}
\newcommand{\p}{\partial}

\author{Henrik Andersen Sveinsson}
\title{Eksamensoppgave 2 - INF5620}
\date{\today}


\begin{document}
\maketitle

\section{Oppgavetekst}

\subsection*{a}
 Set up a wave equation problem in 2D with zero normal derivative as boundary condition. Assume a variable wave velocity.

Mention a physical problem where this mathematical model arises. Explain the physical interpretation of the unknown function.
\subsection*{b}
 Present a finite difference discretization. Explain in particular how the boundary conditions and the initial conditions are incorporated in the scheme.
\subsection*{c}
 Explain (in princple) how the 2D discretization can be extended to 3D.
\subsection*{d}
 Set up the stability condition in 3D. Also quote results on about accuracy of the method in 3D and define the accuracy measure(s) precisely.
\subsection*{e}
 Explain how you can verify the implementation of the method.
\subsection*{f}
 The scheme for the wave equation is perfect for parallel computing. Why? What are the principal ideas behind a parallel version of the scheme? 

\section{Bølgelikningen på grunt vann i 2D}

\begin{equation}
	\frac{\p^2}{\p t^2} u(t, x, y) = \frac{\p}{\p x} \left(q(x, y) \frac{\p}{\p x} u\right) + \frac{\p}{\p y} \left(q(x, y) \frac{\p}{\p y} u\right)
\end{equation}

Denne likningen oppstår ved modellering av tsunamier. Den ukjente funksjonen $u$ er vannivået relativt til gjennomsnittlig vannivå. $q(x, y) = v(x, y)^2$ der $v$ er bølgehastigheten. 


\section{Finite difference disktetisering}

Her velger jeg å bruke tilfellet at $q$ er konstant, for å få litt mindre å skrive. Jeg tenker det er fornuftig når det kun er 15 minutter. 

\end{document}