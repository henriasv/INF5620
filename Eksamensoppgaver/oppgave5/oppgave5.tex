% Eksamensoppgave i INF5620

\documentclass[a4paper, 10pt]{article}
\usepackage[utf8x]{inputenc}
\usepackage{cancel}
\usepackage{graphicx}
\usepackage{amsmath}
\usepackage{geometry}
\newcommand{\mb}{\mathbf}
\newcommand{\mc}{\mathcal}
\newcommand{\n}{\nabla}
\newcommand{\half}{\frac{1}{2}}

\author{Henrik Andersen Sveinsson}
\title{Eksamensoppgave 5 - INF5620}
\date{\today}


\begin{document}
\maketitle

\section{Oppgavetekst}
We look at the PDE problem 
\begin{align*}
u_t &= \nabla\cdot ((1+\alpha_0u^4)\nabla u),\quad & \mb{x} \in\Omega,\
t >0\\
u(\mb{x},0) &= I(\mb{x}),\quad &\mb{x}\in\Omega\\
u(\mb{x},t) &= g(\mb{x}),\quad &\mb{x}\in\partial\Omega_D,\ t>0\\
-(1+\alpha_0u^4)\frac{\partial}{\partial n} u(\mb{x}, t) &= h(u - T_s),\quad
&\mb{x}\in\partial\Omega_R
\end{align*}

Here, u(x,t) is the temperature in some solid material, $I$ is the initial temperature, $g$ is the controlled temperature at a part $\partial \Omega_D$ of the boundary, while at the rest of the boundary, $\partial \Omega_D$, we apply Newton's cooling law with h as a heat transfer coefficient and $T_s$ as the temperature in the surrounding air.

\subsection{a}
Perform a Crank-Nicolson time discretization and define a Picard iteration method for the resulting spatial problems. (Do not pay attention to the boundary conditions.)

\subsection{b} Perform a Backward-Euler time discretization and derive the variational form for the resulting spatial problems. Use a Picard iteration method to linearize the variational form.

\subsection{c} Apply Newton's method to the nonlinear variational form $F=0$ in b). Set up expressions for the right-hand side ($−F$) and the coefficient matrix (Jacobian of $F$) in the linear system that must be solved in each Newton iteration.

\section{Crank-Nicolson i tid og FEM med Picard i rom}
Først skriver vi likningen på generell form:
\begin{equation}
	u_t = \nabla \cdot(\alpha(u)\nabla u)
\end{equation}

Vi begynner med å sette opp en Crank-Nicolson-diskretisering av tidsproblemet:
\begin{equation}
\frac{u^{n+1}- u^n}{\Delta t} = \left[\nabla \cdot (\alpha(u)\nabla u)\right]^{n+\half}
\end{equation}
For å evaluere i $n+\half$ velger vi et aritmetisk snitt av hele høyresiden. Vi innfører samtidig notasjonen $u^{n+1} = u$, $u^n = u_1$.
\begin{equation}
  \frac{u- u_1}{\Delta t} = \half\left[ \nabla \cdot (\alpha(u)\nabla u) + \nabla \cdot (\alpha(u_1) \nabla u_1) \right]
\end{equation}  
Dette leder til 
\begin{equation}
	u - \Delta t\half \nabla \cdot (\alpha(u)\nabla u) = u_1 + \Delta t\half \nabla \cdot (\alpha(u_1)\nabla u_1)
\end{equation}

I denne linkningen er det kun $u$ som er ukjent. Den inngår som et lineært ledd bak nabla-operatoren, men som et ikke-lineært ledd foran. Vi kan bruke picard-iterasjon for å løse dette problemet, og håpe på at løsningen nærmer seg den riktige. 

Picard-iterasjon går ut på at den ukjente i det ikke-lineære leddet "gjettes", i dette tilfellet gjetter vi i utgangspunktet $u = u_1$. Vi løser det romlige problemet med dette satt inn. Vi kaller denne løsningen $u\_$. $u\_$ går nå inn i de ikke-lineære leddene, som en litt bedre tilnærming til $u$ enn den vi hadde. Dette kan gjøres mange ganger, for å komme nærmere den eksakte løsningen av det romlige problemet. 

For ordens skyld skriver vi opp Picard-formuleringen av likningen:
\begin{equation}
	u - \Delta t\half \nabla \cdot (\alpha(u_-)\nabla u) = u_1 + \Delta t\half \nabla \cdot (\alpha(u_1)\nabla u_1)
\end{equation}


\section{Backward Euler og variasjonell formulering}
Vi setter nå opp problemet med Backward Euler. Det gir penere regning, siden vi ikke trenger å snitte i halvstegene. Vi går rett på notasjonen som ligger tettest opp mot programmeringen. 

\begin{equation}
	\frac{u-u_1}{\Delta t} = \nabla \cdot ((1+\alpha_0u^4)\nabla u) 
\end{equation}

Vi setter så alle ledd med $u$ på venstre side

\begin{equation}
	u - \Delta t \nabla ((1+\alpha_0 u^4)\nabla u) = u_1 
\end{equation}

Så lager vi oss en variasjonell formulering:
\begin{equation}
(R, v) = 0
\end{equation}

Det gir:
\begin{equation}
	(u, v) + \Delta t \left[((1+\alpha_0u^4)\nabla u, \nabla v) - 
	\left( (1+\alpha_0 u^4) \frac{\partial u}{\partial n}, v \right)_{\partial \Omega}\right] = (u_1, v)
\end{equation}

Her er det på tide å sette inn randbetingelsen:

\begin{equation}
	-(1+\alpha_0u^4)\frac{\partial}{\partial n} u(\mb{x}, t) = h(u - T_s),\quad
\mb{x}\in\partial\Omega_R
\end{equation}
og 
\begin{equation}
	u(\mb{x}, t) = g(x), \ x\in \Omega_D
\end{equation}
Vi bruker en Boundary-function på den delen av randen som har Dirichlet-betingelse. Det vil si at randbetingelsen er $u=0$ når vi setter opp variasjonsformuleringen. 

Dette gir oss en endelig varisajonell formulering:
\begin{equation}
	(u, v) + \Delta t \left[((1+\alpha_0u^4)\nabla u, \nabla v) + 
	\left( h(u-T_s), v \right)_{\partial \Omega_R}\right] = (u_1, v)
\end{equation}

Så innfører vi picard-iterasjon for å linearisere den variasjonelle formen. Det er i praksis bare å sette inn $u\_$ i alle de ikke-lineære leddene. 


\section{Newtons metode på den ikke-lineære variasjonelle formuleringen}
Det skal finnes en $u$ som oppfyller den variasjonelle formuleringen. Målet med Newtons metode er å finne denne ved å følge gradienten til den variasjonelle formuleringen med hensyn på ekspansjonskoeffisientene i den basisen vi velger å utspenne løsningen i. 

Vi setter opp likningen vi vil løse for newtons metode. Vi setter $(1+\alpha_0 u^4) = \alpha(u)$ for penere notasjon. 
Vi ønsker egentlig å løse
\begin{equation}
	F = 0
\end{equation}
men det klarer vi ikke pga det ikke-lineære leddet. Derfor går vi heller for å løse det for lineariseringen $\hat{F}$ av $F$. Når dette gjøres iterativt kalles det newtons metode. Vi ønsker å justere $u_-$ slik at:
\begin{equation}
	F(u) \approx \hat{F}(u) = F(u_-) + F'(u_-)(u-u_-) = 0
\end{equation}
Det vi effektivt kommer til å løse er
\begin{equation}
	J_{i,j} \delta u = -F(u_-)
\end{equation}
for $\delta u$, og så oppdatere $u_-$. 
\begin{equation}
	(u, v) + \Delta t \left[(\alpha(u)\nabla u, \nabla v) + 
	\left( h(u-T_s), v \right)_{\partial \Omega_R} \right] -(u_1, v) = 0
\end{equation}

Nå antar vi at $u = \sum_{j \in \mc{I}_s} c_j\psi_j$. Da er $v = \psi_i \ \forall i \in \mc{I}_s$.

Vi må regne ut jacobimatrisen til $F$:
\begin{equation}
	J_{i, j} = \frac{\partial F_i}{\partial c_j}
\end{equation}

Vi gjør så derivasjonen med hensyn på $c_i$:
\begin{equation}
	J_{i, j} = (\psi_j, \psi_i) + \Delta t\left[ (\alpha'(u) \psi_j \nabla u, \nabla \psi_i)
	+ (\alpha(u) \nabla \psi_j, \nabla\psi_i)
	+ (h\psi_j, \psi_i)_{\partial \Omega_R} \right]
\end{equation}
$(u_1, v)$ forsvinner siden den bruker $c$-verdier fra et annet tidssteg.

Nå kjenner vi jacobimatrisen. Vi setter inn de kjente verdiene for $u$ fra forrige tidssteg, og bruker dem som utgangspunkt for newtons metode. Vi kaller disse for $u\_$, og kjører metoden til den har konvergert tilstrekkelig bra. I det siste randleddet kjenner vi $u$, siden den er gitt ved $u = g$. Den endelige jacobimatrisen er dermed:
\begin{equation}
		J_{i, j} = (\psi_j, \psi_i) + \Delta t\left[ (\alpha'(u\_) \psi_j \nabla u\_, \nabla \psi_i)
	+ (\alpha(u\_) \nabla \psi_j, \nabla\psi_i)
	+ (h\psi_j, \psi_i)_{\partial \Omega_R}\right]
\end{equation}
\end{document}