\documentclass[a4paper]{article}
\usepackage{amsmath}
%\usepackage{breqn}

\title{Oblig 2 - INF5620}
\author{Henrik Andersen Sveinsson}
\begin{document}
\maketitle

\section{Introducing the problem}
The mathematical problem to be solved is the two-dimensional wave equation:
\begin{equation}
\frac{\partial^2 u}{\partial t^2} + b\frac{\partial u}{\partial t} =
\frac{\partial}{\partial x}\left( q (x,y)
\frac{\partial u}{\partial x}\right) +
\frac{\partial}{\partial y}\left( q (x,y)
\frac{\partial u}{\partial y}\right) + f(x,y,t)
\end{equation}

\section{Deriving the discrete equations}
Inserting finite differences for the derivatives:
\begin{equation}
	[D_tD_t u  + b D_{2t} u = D_x q D_x u + D_y q D_y u + f]_{i, j}^n
\end{equation}
We write this out to prepare it for implementation:
\begin{eqnarray}
	\frac{u_{i, j}^{n+1} - 2u_{i,j}^{n} + u_{i, j}^{n-1}}{\Delta t^2} +
	b\frac{u_{i, j}^{n+1}-u_{i, j}^{n-1}}{2\Delta t} &=& \\
	\frac{q_{i+\frac{1}{2}, j}^n \left(u_{i+1, j}^n - u_{i, j}^n\right) -
	q_{i-\frac{1}{2}, j}^n \left(u_{i, j}^n - u_{i-1, j}^n\right)}{\Delta x^2} \\
	+  \frac{q_{i, j+\frac{1}{2}}^n \left(u_{i, j+1}^n - u_{i, j}^n\right) -
	q_{i, j-\frac{1}{2}}^n \left(u_{i, j}^n - u_{i, j-1}^n\right)}{\Delta y^2} + f_{i, j}^n
\end{eqnarray}
Where the $q_\frac{1}{2}$'s will be evaluated with an arithmetic mean. 
We assume that $\Delta x = \Delta y$ for a simpler scheme. 

Now, we write out the arithmetic means for $q_{\frac{1}{2}}$ and isolate $u_{i, j}^n$. (With sympy)

\begin{eqnarray*}
u_{i, j}^{n+1} &=& \frac{\Delta t^2}{\Delta x^{2} \left(\Delta t b + 2\right)} \left( q^{n}_{{i+1, j}} (u^{n}_{{i+1, j}} - u^{{n}}_{{i, j}}) +  q^{n}_{{i, j+1}} (u^{n}_{{i, j+1}} - u^{{n}}_{{i, j}}) +  q^{n}_{{i, j-1}} (u^{n}_{{i, j-1}} - u^{{n}}_{{i, j}}) \right. \\ &+& \left. q^{n}_{{i, j}} (u^{n}_{{i+1, j}} + u^{n}_{{i, j+1}} + u^{n}_{{i, j-1}} - 4  u^{{n}}_{{i, j}} + u^{n}_{{i-1, j}}) \right. \\
&+& \left.  q^{n}_{{i-1, j}} (-u^{{n}}_{{i, j}} +u^{n}_{{i-1, j}}) + \frac{\Delta x^{2}}{\Delta t^2}\left[ \Delta t b u^{{n-1}}_{{i, j}} - 2  u^{{n-1}}_{{i, j}} + 4  u^{{n}}_{{i, j}} \right] \right)
\end{eqnarray*}

\section{Domain and boundary condition}
The partial differential equation is to be solved on a domain $\Omega = [0, L_x] \times [0, L_y]$, which will be discretized by a mesh $(i, j)$ where $i \in (0, N_x)$ and $j \in (0, N_y)$.
The boundary condition is $\frac{\partial u}{\partial n} = 0$ where $n \in \{x, y\}$. The numerical scheme as it is formulated now, will required to evaluate points outside the mesh in order to calculate the soultion on the boundary. However, with a symmetric nearest neighbor finite difference, the boundary condition requires $u_{i, j}^n = u_{i+1, j}$ for the boundary at $x = L_x$, similarly for the other boundaries.
In the implementation, these points outsie the mesh are implmemented as ghost cells, and they are updated after all other calculations in that timestep.  

\end{document}