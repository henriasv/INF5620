\documentclass[a4paper]{article}
\usepackage{amsmath}
%\usepackage{breqn}

\title{Oblig 2 - INF5620}
\author{Henrik Andersen Sveinsson}
\begin{document}
\maketitle

\section{Introducing the problem}
The mathematical problem to be solved is the two-dimensional wave equation:
\begin{equation}
\frac{\partial^2 u}{\partial t^2} + b\frac{\partial u}{\partial t} =
\frac{\partial}{\partial x}\left( q (x,y)
\frac{\partial u}{\partial x}\right) +
\frac{\partial}{\partial y}\left( q (x,y)
\frac{\partial u}{\partial y}\right) + f(x,y,t)
\end{equation}

\section{Deriving the discrete equations}
Inserting finite differences for the derivatives:
\begin{equation}
[D_tD_t u + b D_{2t} u = D_x q D_x u + D_y q D_y u + f]_{i, j}^n
\end{equation}
We write this out to prepare it for implementation:
\begin{eqnarray}
\frac{u_{i, j}^{n+1} - 2u_{i,j}^{n} + u_{i, j}^{n-1}}{\Delta t^2} +
b\frac{u_{i, j}^{n+1}-u_{i, j}^{n-1}}{2\Delta t} &=& \\
\frac{q_{i+\frac{1}{2}, j}^n \left(u_{i+1, j}^n - u_{i, j}^n\right) -
q_{i-\frac{1}{2}, j}^n \left(u_{i, j}^n - u_{i-1, j}^n\right)}{\Delta x^2} \\
+ \frac{q_{i, j+\frac{1}{2}}^n \left(u_{i, j+1}^n - u_{i, j}^n\right) -
q_{i, j-\frac{1}{2}}^n \left(u_{i, j}^n - u_{i, j-1}^n\right)}{\Delta y^2} + f_{i, j}^n
\end{eqnarray}
Where the $q_\frac{1}{2}$'s will be evaluated with an arithmetic mean.
We assume that $\Delta x = \Delta y$ for a simpler scheme.

Now, we write out the arithmetic means for $q_{\frac{1}{2}}$ and isolate $u_{i, j}^n$. (With sympy)

\begin{eqnarray*}
u_{i, j}^{n+1} &=& \frac{\Delta t^2}{\Delta x^{2} \left(\Delta t b + 2\right)} \left( q^{n}_{{i+1, j}} (u^{n}_{{i+1, j}} - u^{{n}}_{{i, j}}) + q^{n}_{{i, j+1}} (u^{n}_{{i, j+1}} - u^{{n}}_{{i, j}}) + q^{n}_{{i, j-1}} (u^{n}_{{i, j-1}} - u^{{n}}_{{i, j}}) \right. \\ &+& \left. q^{n}_{{i, j}} (u^{n}_{{i+1, j}} + u^{n}_{{i, j+1}} + u^{n}_{{i, j-1}} - 4 u^{{n}}_{{i, j}} + u^{n}_{{i-1, j}}) \right. \\
&+& \left. q^{n}_{{i-1, j}} (-u^{{n}}_{{i, j}} +u^{n}_{{i-1, j}}) + \frac{\Delta x^{2}}{\Delta t^2}\left[ \Delta t b u^{{n-1}}_{{i, j}} - 2 u^{{n-1}}_{{i, j}} + 4 u^{{n}}_{{i, j}} \right] \right)
\end{eqnarray*}

\section{Domain and boundary condition}
The partial differential equation is to be solved on a domain $\Omega = [0, L_x] \times [0, L_y]$, which will be discretized by a mesh $(i, j)$ where $i \in (0, N_x)$ and $j \in (0, N_y)$.
The boundary condition is $\frac{\partial u}{\partial n} = 0$ where $n \in \{x, y\}$. The numerical scheme as it is formulated now, will required to evaluate points outside the mesh in order to calculate the soultion on the boundary. However, with a symmetric nearest neighbor finite difference, the boundary condition requires $u_{i, j}^n = u_{i+1, j}$ for the boundary at $x = L_x$, similarly for the other boundaries.
In the implementation, these points outsie the mesh are implmemented as ghost cells, and they are updated after all other calculations in that timestep.

\section{Initial conditions and special formula for the first timestep}

$$\frac{\Delta t^{2} \left(2 \Delta x^{2} f + q^{n}_{{i+1, j}} u^{n}_{{i+1, j}} - q^{n}_{{i+1, j}} u^{{n}}_{{i, j}} + q^{n}_{{i, j+1}} u^{n}_{{i, j+1}} - q^{n}_{{i, j+1}} u^{{n}}_{{i, j}} + q^{n}_{{i, j-1}} u^{n}_{{i, j-1}} - q^{n}_{{i, j-1}} u^{{n}}_{{i, j}} + q^{n}_{{i, j}} u^{n}_{{i+1, j}} + q^{n}_{{i, j}} u^{n}_{{i, j+1}} + q^{n}_{{i, j}} u^{n}_{{i, j-1}} - 4 q^{n}_{{i, j}} u^{{n}}_{{i, j}} + q^{n}_{{i, j}} u^{n}_{{i-1, j}} - q^{n}_{{i-1, j}} u^{{n}}_{{i, j}} + q^{n}_{{i-1, j}} u^{n}_{{i-1, j}}\right) + \Delta t \Delta x^{2} b u^{{n-1}}_{{i, j}} - 2 \Delta x^{2} u^{{n-1}}_{{i, j}} + 4 \Delta x^{2} u^{{n}}_{{i, j}}}{\Delta x^{2} \left(\Delta t b + 2\right)}
$$

$$\frac{\Delta t^{2} \left(2 \Delta x^{2} f + q^{n}_{{i+1, j}} u^{n}_{{i+1, j}} - q^{n}_{{i+1, j}} u^{{n}}_{{i, j}} + q^{n}_{{i, j+1}} u^{n}_{{i, j+1}} - q^{n}_{{i, j+1}} u^{{n}}_{{i, j}} + q^{n}_{{i, j-1}} u^{n}_{{i, j-1}} - q^{n}_{{i, j-1}} u^{{n}}_{{i, j}} + q^{n}_{{i, j}} u^{n}_{{i+1, j}} + q^{n}_{{i, j}} u^{n}_{{i, j+1}} + q^{n}_{{i, j}} u^{n}_{{i, j-1}} - 4 q^{n}_{{i, j}} u^{{n}}_{{i, j}} + q^{n}_{{i, j}} u^{n}_{{i-1, j}} - q^{n}_{{i-1, j}} u^{{n}}_{{i, j}} + q^{n}_{{i-1, j}} u^{n}_{{i-1, j}}\right) + 4 \Delta x^{2} u^{{n}}_{{i, j}}}{4 \Delta x^{2}}$$

There are certain restrictions on the initial conditions. For example, there should be no overall drift in the solution. 

\section{Truncation error}
The truncation error is alwas defined as the error that arises when the exact solution to an equation is inserted into a scheme to solve that equation.

In our case that means that the error is $R$ in the following equation:
\begin{equation}
	[D_tD_t u_e + bD_tu_e = D_xqD_xu_e + D_yqD_yu_e + f + R]_{i,j}^n
\end{equation}
First, we show the truncation error with constant q, note that in the following equation, all $u$'s are the exact solution.
\begin{eqnarray*}
R^n = u_{tt}(t_n) + \frac{1}{12}u_{ttt}(t_n) \Delta t^2 + \mathcal{O}(\Delta t^4)
+ b(u_t(t_n) + \frac{1}{6}u_{ttt}(t_n) \Delta t^2) \\
- q(u_{xx}(t_n) + \frac{1}{12}u_{xxx}\Delta x^2 + \mathcal{O}(\Delta x^4))
- q(u_{yy}(t_n) + \frac{1}{12}u_{yyy}\Delta y^2 + \mathcal{O}(\Delta y^4))
- f(t_n)
\end{eqnarray*}
Since $u$ is indeed the exact solution, we can subtract the initial equation, such that:
\begin{equation}
R^n = \left(\frac{1}{12}u_{ttt}(t_n) + \frac{1}{6}bu_{ttt}(t_n)\right)\Delta t^2 
+ \frac{1}{12}q\left(u_{xxx}\Delta x^2 + u_{yyy}\Delta y^2 \right) + \mathcal{O}(\Delta t^4, \Delta x^4, \Delta y^4)
\end{equation}
So, the truncation error is of second order both in time and space. 

Then, for a non-constant $q$, we are interested in the difference between $[D_xq(x, y)D_x u_e]$ and $\frac{\partial}{\partial x} q(x, y) \frac{\partial u}{\partial x}$. In order to find this out, we make a taylor series of all the terms involved in the numerical scheme for this expression. This is a long and tedious computation, but it is done for this exact expression in the course material, and the error is of second order.

Therefore the whole scheme is of second order. 

\section{Verification}
\subsection{Constant solution}
We construct a constant solution:
\begin{equation}
u(t, x, y) = C
\end{equation}
This is the trivial solution to the equation, and requires $f(t, x, y) = 0$ and $V(x, y) = 0$. 

\subsubsection{What if the mathematical formulas were incorrectly implemented}
\begin{itemize}
\item $(b\Delta t +2)$ omitted in the first denominator
\item wrong indices in $u_{i, j}^{n+1}$ (See code)
\end{itemize}

\section{Plug wave solution}
An initial condition on the form:
\begin{equation}
I(x, y) = 
\begin{cases}
\text{constant} & \text{ if } x\in (L/2\pm\epsilon) \\
0 & \text{ otherwise}
\end{cases}	
\end{equation}
Where $\epsilon$ is som e finite number $< L/2$.


\section{Standing undamped waves}
This test is not trivial. The numerical solution will oscillate out of sync with the exact one, causing th solution to be off with $\approx 2A$ at most. I guess the point of this exercise is to see how fast this happens, as a value in timesteps. 

\end{document}